\documentclass{bioinfo}
\copyrightyear{2011}
\pubyear{2011}

\usepackage{amsthm}
\usepackage{natbib}
\bibliographystyle{apalike}

\begin{document}
\newtheorem{thm}{Theorem}[section]
\newtheorem{lem}[thm]{Lemma}
\newtheorem{proposition}[thm]{Proposition}
\firstpage{1}

\title[FM-index]{Mathematical Notes on FM-index}

\author[Li]{Heng Li$^1$}

\address{$^1$Broad Institute, 7 Cambridge Center, Cambridge, MA 02142}

\maketitle

\begin{methods}
\section{METHODS}
\subsection{Preliminaries}

\begin{table}[!htb]
\processtable{Notations}
{\begin{tabular}{lp{7cm}}
\toprule
Symbol & Description \\
\midrule
$X$ & String: $X=a_0a_1\ldots a_{n-1}$ with $a_{n-1}=\$$\\
$X[i]$ & The $i$-th symbol in string $X$: $X[i]=a_i$\\
$X[i,j]$ & Substring: $X[i,j]=a_i\ldots a_j$\\
$X_i$ & Suffix: $X_i=X[i,n-1]$\\
$\mathcal{S}(X)$ & Set of suffixes: $\mathcal{S}=\{X_i:0\le i\le n-1\}$\\
$S'$ & Inverse suffix array: $S'(i)=|\{j:X_j<X_i\}|$\\
$S$ & Suffix array, the inverse of $S'$ (i.e. $S'(S(i))=S(S'(i))=i$)\\
$B$ & BWT: $B[i]=X[S(i)-1]$ if $S(i)>0$ or $B[i]=\$$ otherwise\\
$\Psi$ & CSA: $\Psi(i)=S'(S(i)+1)$\\
$\Psi'$ & Inverse CSA: $\Psi'(i)=S'(S(i)-1)$\\
$C(a)$ & $C(a)=|\{0\le i\le n-1:X[i]<a\}|$ \\
$r(a,i)$ & $r(a,i)=|\{0\le j\le i:B[j]=a\}|$\\
$s(a,k)$ & $s(a,k)=i$ such that $r(a,i)=k$ and $r(a,i+1)=k+1$\\
\botrule
\end{tabular}}{}
\end{table}

\subsubsection{Strings}
Let $\Sigma$ be a finite alphabet. Symbol `\$', which is called a
\emph{sentinel}, is not present in $\Sigma$ and is lexicographically smaller
than all symbols in $\Sigma$. Given a string $X=a_0a_1\ldots a_{n-1}$ over
$\Sigma\cup\{\$\}$, let $X[i]=a_i$ be the symbol at position $i$ ($i=0,1,\ldots,n-1$),
$X[i,j]=a_i\ldots a_j$ be a substring from position $i$ to $j$ inclusive, and
$X_i=X[i,n-1]$ be the suffix starting from position $i$.  The length of a
string $X$, denoted by $|X|$, equals the number of symbols in $X$.  In
particular, a string that does not contain any symbols is called a \emph{null
string} and is written as $\epsilon$. Obviously, $|\epsilon|=0$.

\subsubsection{Sentinels}
A string $X$ is said to be \emph{terminated} if its last symbol is a sentinel
(i.e. $X[|X|-1]=\$$). Different from many other literatures, we allow a string
to contain multiple sentinels and a sentinel to appear in the middle of the
string. The lexicographical order of two sentinels is determined by their
positions in the string. More exactly, suppose both $X[i]$ and $X[j]$ are
sentinels.  $X[i]<X[j]$ stands if and only if $i<j$.

\subsubsection{Suffix array (SA)}
Suppose $X$ is a terminated string. We start from the \emph{inverse suffix
array}, which is defined on $\{0,\ldots,|X|-1\}$ as
\begin{equation}\label{eq:iS}
S'(i)\triangleq|\{j:X_j<X_i\}|
\end{equation}
i.e. $S'(i)$ is the rank of suffix $X_i$.
Obviously, $0\leq S'(i)\leq |X|-1$, and $S'(i)=S'(j)$ if and only if $i=j$.
Thus $S'$ is in fact a permutation of integers $0,\ldots,|X|-1$ and invertible.
The inverse of $S'$ is \emph{suffix array} $S$. In a suffix array $S$, $S(k)$
is the position of the $k$-th ($k=0,\ldots,|X|-1$) smallest suffix.

\subsubsection{Compressed suffix array (CSA)}
The compressed suffix array (CSA) $\Psi$ is defined as
\begin{equation}\label{eq:Psi}
\Psi(k)\triangleq\left\{\begin{array}{ll}
S'(S(k)-1) & \mbox{if $S(k)>0$}\\
S'(|X|-1) & \mbox{otherwise}
\end{array}\right.
\end{equation}
CSA $\Psi$ is also a permutation of integers $0,\ldots,|X|-1$ and thus invertible.
We can prove that the inverse CSA $\Psi'$ satisfies that
\begin{equation}\label{eq:iPsi}
\Psi'(k)\triangleq\left\{\begin{array}{ll}
S'(S(k)+1) & \mbox{if $S(k)\not=|X|-1$}\\
S'(0) & \mbox{otherwise}
\end{array}\right.
\end{equation}

%\begin{table}[!htb]
%\processtable{Suffix array, BWT and CSA of $X=\mathtt{GTG\$}_1\mathtt{ATG\$}_2$}
%{\begin{tabular}{|ccccc|lc|c|}
%\hline
%$i$ & $S(i)$ & $S'(i)$ & $\Psi(i)$ & $\Psi'(i)$ & & $B[i]$ & {\tt \$AGT}\\
%\hline
%0 & 3 & 5 & 2& 3& {\tt 1ATG2GT} & {\tt G} & {\tt 0010} \\
%1 & 7 & 6 & 5& 4& {\tt 2GTG1AT} & {\tt G} & {\tt 0020} \\
%2 & 4 & 3 & 7& 0& {\tt ATG2GTG} & {\tt \$}$_{1}$& {\tt 1020} \\
%3 & 2 & 0 & 0& 6& {\tt G1ATG2G} & {\tt T} & {\tt 1021} \\
%4 & 6 & 2 & 1& 7& {\tt G2GTG1A} & {\tt T} & {\tt 1022} \\
%5 & 0 & 7 & 6& 1& {\tt GTG1ATG} & {\tt \$}$_{2}$& {\tt 2022} \\
%6 & 1 & 4 & 3& 5& {\tt TG1ATG2} & {\tt G} & {\tt 2032} \\
%7 & 5 & 1 & 4& 2& {\tt TG2GTG1} & {\tt A} & {\tt 2132} \\
%\hline
%\end{tabular}}{}
%\end{table}

\begin{table}[!htb]
\processtable{Suffix array, BWT and CSA of $X=\mathtt{\$AG\$CT\$AC\$GT\$}$}
{\begin{tabular}{|rl|rr|ll|c|}
\hline
$i$ & $X[i]$ & $S(i)$ & $\Psi'(i)$ & Suffix & $B[i]$ & {\tt \$ACGT}\\
\hline
0 &{\tt\$}$_0$ &0 &0 & {\tt \$}$_0$ & {\tt \$}$_4$ & {\tt 10000} \\
1 &{\tt A}     &3 &9 & {\tt \$}$_1$ & {\tt G}      & {\tt 10010} \\
2 &{\tt G}     &6 &11& {\tt \$}$_2$ & {\tt T}      & {\tt 10011} \\
3 &{\tt\$}$_1$ &9 &7 & {\tt \$}$_3$ & {\tt C}      & {\tt 10111} \\
4 &{\tt C}     &12&12& {\tt \$}$_4$ & {\tt T}      & {\tt 10112} \\
5 &{\tt T}     &7 &1 & {\tt AC\$}$_3$&{\tt \$}$_2$ & {\tt 20112} \\
6 &{\tt\$}$_2$ &1 &2 & {\tt AG\$}$_1$&{\tt \$}$_0$ & {\tt 30112} \\
7 &{\tt A}     &8 &5 & {\tt C\$}$_3$& {\tt A}      & {\tt 31112} \\
8 &{\tt C}     &4 &3 & {\tt CT\$}$_2$&{\tt \$}$_1$ & {\tt 41112} \\
9 &{\tt\$}$_3$ &2 &6 & {\tt G\$}$_1$& {\tt A}      & {\tt 42112} \\
10&{\tt G}     &10&4 & {\tt GT\$}$_4$&{\tt \$}$_3$ & {\tt 52112} \\
11&{\tt T}     &5 &8 & {\tt T\$}$_2$& {\tt C}      & {\tt 52212} \\
12&{\tt\$}$_4$ &11&10& {\tt T\$}$_4$& {\tt G}      & {\tt 52222} \\
\hline
\end{tabular}}{}
\end{table}

\begin{thm}[Inverse CSA]
\begin{equation}\label{eq:psii}
\Psi'(i)=C(B[i])+r(B[i],i)-1
\end{equation}
\end{thm}
\begin{proof}
By definition:
\begin{eqnarray*}
C(B[i])&=&\big|\{j:B[j]<B[i]\}\big|\\
&=&\big|\{j:X[S(j)-1]<X[S(i)-1]\}\big|\\
&=&\big|\{l\le1:X[l-1]<X[k-1]\land k=S(i)\}\big|
\end{eqnarray*}
and
\begin{eqnarray*}
&&r(B[i],i)\\
&=&\big|\{j:B[i]=B[j]\land j\le i\}\big|\\
&=&\big|\{j:X[S(i)-1]=X[S(j)-1]\land j\le i\}\big|\\
&=&\big|\{l:X[l-1]=X[k-1]\land S'(l)\le S'(k)\land k=S(i)\}\big|\\
&=&\big|\{l:X[l-1]=X[k-1]\land X_l\le X_k\land k=S(i)\}\big|\\
&=&\big|\{l:X[l-1]=X[k-1]\land X_{l-1}\le X_{k-1}\land k=S(i)\}\big|\\
\end{eqnarray*}
Therefore
\begin{eqnarray*}
&&C(B[i])+r(B[i],i)-1\\
&=&\big|\{X_{l-1}\le X_{k-1}\land k=S(i)\}\big|-1\\
&=&S'(S(i)-1)\\
&=&\Psi'(i)
\end{eqnarray*}
\begin{flushright}\qedsymbol\end{flushright}
\end{proof}

\begin{lem}[Backward search]
Let $\flat$ and $\sharp$ are two symbols s.t. $\flat<a$ and $\sharp>a$ for all $a\in\Sigma\cup\{\$\}$.
Given a string $W$, define:
\begin{eqnarray*}
R_L(W)&=&\big|\{i:X_i<W\flat\}\big| + 1 \\
R_U(W)&=&\big|\{i:X_i<W\sharp\}\big|
\end{eqnarray*}
Then:
\begin{eqnarray}
R_L(aW)&=&C(a)+r(a,R_L(W)-1)\\
R_U(aW)&=&C(a)+r(a,R_U(W))-1
\end{eqnarray}
\end{lem}
\begin{proof}
By definition,
$$C(a)=\big|\{l:X[l-1]<a\}\big|$$
and similarly follow the derivation of Eq.~\eqref{eq:psii}
\begin{eqnarray*}
r(a,R_L(W)-1)&=&\big|\{l:X[l-1]=a\land S'(l)<R_L(W)\}\big|\\
&=&\big|\{l:X[l-1]=a\land X_l<W\flat\}\big|\\
&=&\big|\{l:X[l-1]=a\land X_{l-1}<aW\flat\}\big|\\
\end{eqnarray*}
Therefore
$$
C(a)+r(a,R_L(W)-1)=\big|\{l:X_{l-1}<aW\flat\}\big|=R_L(aW)
$$
Similarly
\begin{eqnarray*}
r(a,R_U(W))&=&\big|\{l:X[l-1]=a\land S'(l)\le R_U(W)\}\big|\\
&=&\big|\{l:X[l-1]=a\land S'(l)<R_U(W)\}\big|+1\\
&=&\big|\{l:X[l-1]=a\land X_l<W\flat\}\big|+1\\
&=&\big|\{l:X[l-1]=a\land X_{l-1}<aW\flat\}\big|+1\\
\end{eqnarray*}
The other equation can be proved.

\raggedleft{\qedsymbol}
\end{proof}

\begin{lem}[Forward search]
\begin{eqnarray*}
R_L(Wa)&=&R_L(W)+\sum_{b<a}\Big[R_U'(bW')-R_L'(bW')\Big]\\
R_U(Wa)&=&R_L(W)+\sum_{b\le a}\Big[R_U'(bW')-R_L'(bW')\Big]\\
\end{eqnarray*}
\end{lem}
\begin{proof}
We note that
$$
R_U(W)-R_L(W)=R'_U(W')-R'_L(W')
$$
and
$$
R_L(Wa)=R_L(W)+\sum_{b<a}\Big[R_U(Wb)-R_L(Wb)\Big]
$$
The two equations can be easily proved.

\raggedleft{\qedsymbol}
\end{proof}

\end{methods}

\end{document}
